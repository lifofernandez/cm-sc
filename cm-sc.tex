%% 
%% Copyright 2019-2021 Elsevier Ltd
%% 
%% This file is part of the 'CAS Bundle'.
%% --------------------------------------
%% 
%% It may be distributed under the conditions of the LaTeX Project Public
%% License, either version 1.2 of this license or (at your option) any
%% later version.  The latest version of this license is in
%%    http://www.latex-project.org/lppl.txt
%% and version 1.2 or later is part of all distributions of LaTeX
%% version 1999/12/01 or later.
%% 
%% The list of all files belonging to the 'CAS Bundle' is
%% given in the file `manifest.txt'.
%% 
%% Template article for cas-dc documentclass for 
%% double column output.

\documentclass[a4paper,fleqn,spanish]{cas-dc}
% corte de palabras en español
\usepackage[spanish]{babel}
%\usepackage[style=numeric-comp]{biblatex}

%\renewcommand{\abstractname}{Abstract} 
%\renewcommand{\abstract}{Abstract} 
\addto{\captionsspanish}{
  \renewcommand{\abstractname}{ABSTRACT}
}

% If the frontmatter runs over more than one page
% use the longmktitle option.

%\documentclass[a4paper,fleqn,longmktitle]{cas-dc}

%\usepackage[numbers]{natbib}
%\usepackage[authoryear]{natbib}
\usepackage[authoryear,longnamesfirst]{natbib}

%\usepackage{lipsum}

% Borrar cover page
\usepackage{atbegshi}% http://ctan.org/pkg/atbegshi
\AtBeginDocument{\AtBeginShipoutNext{\AtBeginShipoutDiscard}}

%%%Author macros
\def\tsc#1{\csdef{#1}{\textsc{\lowercase{#1}}\xspace}}
\tsc{WGM}
\tsc{QE}
%%%

% Uncomment and use as if needed
%\newtheorem{theorem}{Theorem}
%\newtheorem{lemma}[theorem]{Lemma}
%\newdefinition{rmk}{Remark}
%\newproof{pf}{Proof}
%\newproof{pot}{Proof of Theorem \ref{thm}}

\renewcommand{\refname}{Referencias}

\begin{document}

\let\WriteBookmarks\relax
\def\floatpagepagefraction{1}
\def\textpagefraction{.001}

% Short title
\shorttitle{KD in the ESN}    

% Short author
\shortauthors{J. Cerviño, J. L. Gobbe y L. Fernández}  

% Main title of the paper
%\title[mode = title]{Knowledge Discovery in the Enterprise Social Network}    
%\title[mode = title]{Trabajo Práctico Final: Informe del Seminario Integrador}    

\title[mode = title]{Modelado Conceptual de Smart Cities}    

% Title footnote mark
% eg: \tnotemark[1]
%\tnotemark[1] 
%
%% Title footnote 1.
%% eg: \tnotetext[1]{Title footnote text}
%\tnotetext[1]{Nota al pie 1} 

% First author
%
% Options: Use if required
% eg: \author[1,3]{Author Name}[type=editor,
%       style=chinese,
%       auid=000,
%       bioid=1,
%       prefix=Sir,
%       orcid=0000-0000-0000-0000,
%       facebook=<facebook id>,
%       twitter=<twitter id>,
%       linkedin=<linkedin id>,
%       gplus=<gplus id>]

%\author[1]{<author name>}[<options>]

\author[1]{Joaquín Cerviño}[
  type=editor,
  style=spanish,
  auid=000,
  bioid=1,
  prefix=Lic.,
  %orcid=0000-0000-0000-0000,
  %facebook=<facebook id>,
  %twitter=<twitter id>,
  %linkedin=<linkedin id>,
  %gplus=<gplus id>
]

% Corresponding author indication
%\cormark[<corr mark no>]
\cormark[1]

% Footnote of the first author
%\fnmark[<footnote mark no>]
%\fnmark[1]

% Email id of the first author
\ead{cjoackin@gmail.com}

% URL of the first author
\ead[https://ioadeer.github.io]{https://ioadeer.github.io}

% Credit authorship
%\credit{Credit authorship details}
% eg: \credit{Conceptualization of this study, Methodology, Software}
\credit{Conceptualization of this study, Methodology, Writing}

\author[2]{José Luis Gobbe}[
  type=editor,
  style=spanish,
  auid=001,
  bioid=2,
  prefix=Ing.,
]
\author[3]{Lisandro Fernández}[
  type=editor,
  style=spanish,
  auid=002,
  bioid=3,
  prefix=Lic.,
]

% Address/affiliation
\affiliation[1-3]{
  organization={
    Universidad Tecnológica Nacional,
    Facultad Regional Buenos Aires
},
  addressline={Medrano 951}, 
  city={Buenos Aires},
  % citysep={}, % Uncomment if no comma needed between city and postcode
  postcode={C1179AAQ}, 
  state={C.A.B.A},
  country={Argentina}
}


% \author[2]{<author name>}[<options>]
% 
% % Footnote of the second author
% \fnmark[2]
% 
% % Email id of the second author
% \ead{}
% 
% % URL of the second author
% \ead[url]{}
% 
% % Credit authorship
% \credit{}
% 
% % Address/affiliation
% \affiliation[<aff no>]{organization={},
%             addressline={}, 
%             city={},
% %          citysep={}, % Uncomment if no comma needed between city and postcode
%             postcode={}, 
%             state={},
%             country={}}
% 

% Corresponding author text
\cortext[1]{Corresponding author}

% Footnote text
% \fntext[1]{Texto nota al pie autor}

% For a title note without a number/mark
\nonumnote{}

\selectlanguage{english}
% Here goes the abstract
\begin{abstract}
The constant increase in the use of the Enterprise Social Network (ESN), as a
channel for organizational collaboration and the large volume of data produced,
provides exceptional opportunities for decision-making in management. The study
of this phenomenon is proposed in order to identify patterns of technological
integration, data structures, techniques, and manipulation mechanisms involved
in the Knowledge Discovery (KD) process in ESNs.
\end{abstract}
\selectlanguage{spanish}

% Use if graphical abstract is present
%\begin{graphicalabstract}
%\includegraphics{}
%\end{graphicalabstract}

% Research highlights
\begin{highlights}
\item pepe
\item pepe
\item pepe
\end{highlights}

%Trabajo Práctico Final: INFORME DEL SEMINARIO INTEGRADOR

% Keywords
% Each keyword is seperated by \sep
\begin{keywords}
Conceptual Model\sep Smart Cities
%\sep
%pepe\sep
\end{keywords}
\maketitle
% Main text
\section{Proyecto Propuesta Trabajo Final de la Especialización}\label{TFE}

Se propone el tema principal para el Trabajo Final de la Especialización en
Ingeniería en Sistemas de Información de la Universidad Tecnológica Nacional
Facultad Regional Buenos Aires (UTN.BA).

%\subsection{Título tentativo}\label{titulo-TFE}
\subsection{Descubrimiento del Conocimiento en la Red Social Empresarial - Un estudio de mapeo sistemático}\label{titulo-TFE}
%\subsection{Objetivo del Trabajo Final de Especialidad}\label{objetivo-TFE}

El presente proyecto de Trabajo Final de Especialidad propone el estudio de la
Red Social Empresarial (Enterprise Social Network - ESN) con el objetivo de
extraer información, su contextualización y refinamiento para el Descubrimiento
del Conocimiento (Knowledge Discovery - KD).

La propuesta de TFE es el desarrollo de un Estudio de Mapeo Sistemático
(Systematic Mapping Study - SMS) de la producción académica internacional sobre
los aspectos considerados al integrar tecnologías en las redes sociales desde
el enfoque organizacional y el Descubrimiento del Conocimiento, revisar
modelos, roles y comportamiento de agentes involucrados, creencias compartidas,
normas, patrones de refinamiento y marcos de validación transversal.

El tipo de revisión de literatura elegido es SMS dado que fomenta la
disposición de las evidencias en el dominio con alto nivel de granularidad,
ayuda a identificar grupos de estudios afines y evidencia áreas de vacancia
vigentes para investigaciones futuras.


\section{Proyecto Propuesta Tesis de Maestría}\label{tesis}

En esta sección se presenta el objeto de estudio de la Tesis de la Maestría en
Ingeniería en Sistemas de Información de UTN.BA. Se exponen las razones que
justfican el estudio y se coteja con los antecedentes considerados hasta el
momento.

%\subsection{Título tentativo}\label{titulo-TESIS}
\subsection{Descubrimiento del Conocimiento en la Red Social Empresarial}\label{titulo-TESIS}

%\subsection{Descripción del problema - Identificación de Objeto}\label{que}

El incremento constante en el uso de la Red Social Empresarial (Enterprise
Social Network - ESN) \cite{Wehner_2017} como canal de colaboración
organizacional y el gran volumen de datos producido, brinda oportunidades
excepcionales de estudio para el análisis y la toma de decisiones en la gestión
\cite{Schwade2018}.

El proceso del Descubrimiento del Conocimiento (Knowledge Discovery - KD) en la
ESN trata de extraer patrones en forma de reglas o funciones, a partir de los
datos implicados en la creación e intercambio del contenido generado por el
usuario \cite{Tao2020}. Esta tarea involucra analizar y derivar información
nueva de estos aportes, por medio de la identificación de correlaciones entre
términos, logrando explicitar conocimiento mediante la compresión del registro
de eventos que producen los sistemas de colaboración de la organización
\cite{OLeary2017}.

El interés de esta investigación son los patrones de integración tecnológica,
estructuras de datos, técnicas y mecanismos de manipulación involucrados en el
proceso de extraer conocimiento en las ESN \cite{Hacker2017,Schwade2018}. Este
fenómeno, que se manifiesta en las comunidades online de las organizaciones, es
complejo, multifacético y orbita las tareas de: desarrollar, monitorear y
mejorar procesos; describir objetivos, identificar variables implicadas y
determinar constricciones o posibles alternativas de decisión.

Este estudio versa sobre cómo comprender un gran lote de datos, una actividad
relacionada con diferentes disciplinas: Recuperación de la Información
(Information Retrieval - IR), que es el proceso de obtener información
relevante de un sistema a partir de la recolección de muestras de los recursos
del mismo. Los datos pueden ser explorados a través del Aprendizaje Automático
(Machine Learning - ML), el subcampo de la Inteligencia Artificial cuyo
compendio de algoritmos informáticos pueden evolucionar automáticamente a
través de la experiencia. Análisis predictivo, del análisis estadístico
aplicado a los problemas gerenciales, que es encontrar tendencias, patrones y
relaciones utilizando datos cuantitativos. Comprender el significado semántico
del contenido en los documentos para extraer información y organizarlos implica
técnicas de Procesamiento del Lenguaje Natural (Natural Language Processing -
NLP).

La acumulación de datos, producto de la transferencia y creación de contenido,
dentro de la ESN proporciona una fuente sustancial de noticias cruciales que
pueden ser sometidas al KD. 

% Numbered list
% Use the style of numbering in square brackets.
% If nothing is used, default style will be taken.
%\begin{enumerate}[a)]
%\item 
%\item 
%\item 
%\end{enumerate}  

% Unnumbered list
%\begin{itemize}
%\item 
%\item 
%\item 
%\end{itemize}  

% Description list
%\begin{description}
%\item[]
%\item[] 
%\item[] 
%\end{description}  

%% % Figure
%% \begin{figure}[<options>]
%% 	\centering
%% 		\includegraphics[<options>]{}
%% 	  \caption{}\label{fig1}
%% \end{figure}
%% 
%% 
%% \begin{table}[<options>]
%% \caption{}\label{tbl1}
%% \begin{tabular*}{\tblwidth}{@{}LL@{}}
%% \toprule
%%   &  \\ % Table header row
%% \midrule
%%  & \\
%%  & \\
%%  & \\
%%  & \\
%% \bottomrule
%% \end{tabular*}
%% \end{table}

% Uncomment and use as the case may be
%\begin{theorem} 
%\end{theorem}

% Uncomment and use as the case may be
%\begin{lemma} 
%\end{lemma}

%% The Appendices part is started with the command \appendix;
%% appendix sections are then done as normal sections
%% \appendix

%\section{pepe2 }\label{Pepe2}

\subsection{¿Por qué es importante resolverlo?}\label{porque}

La disponibilidad, facilidad de uso y escalabilidad de sistemas lo
suficientemente robustos para manejar instancias a gran escala de registro de
datos estructurados y no estructurados de múltiples fuentes ha creado entornos
de trabajo fácil, colaborativo y distributivo.

De la necesidad de mejorar a partir del proceso de intercambio, surge el
desafío del volumen en constante aumento de los datos que se generan. El reto
al que se enfrentan las organizaciones es la pericia de obtener provecho de
toda esa información para identificar patrones y anomalías, diagnosticar
problemas, señalar áreas y prescribir tareas de mejora \cite{Qi_2017}.

La proliferación en el intercambio, la democratización de técnicas y procesos
de análisis da origen a una generación de tecnologías y arquitecturas nuevas.
Dicho proceso puede ser analizado para conseguir elementos explícitos que
permitan a las organizaciones comprender y mejorar sus procesos. Siendo que los
registros de la ESN no solo contienen la amplitud de noticias necesarias para
entender el mismo componente, sino que incluye información retrospectiva del
intercambio disponible para preparar estudios y evaluar hipótesis, aportando
valor agregado al registro.

\subsection{¿Qué han hecho otros y por qué la propuesta es superadora?}\label{otros}

El proceso de extraer conocimiento de grandes volúmenes de datos es reconocido
en la comunidad de investigadores como tópico clave en los sistemas de
información y por muchas empresas comerciales significa una oportunidad para
obtener ganancias \cite{Ahmed2017,Centobelli_2017,Cetto2018}. Esto ha sido
aprovechado en diferentes ámbitos de la sociedad, en diversos tipos de
sistemas, en este caso la ESN como pieza central de las comunidades online
\cite{Engelbrecht2019,Hao2021,Nisar_2019}.

Si bien muchos estudios empíricos y elaboraciones teóricas aportaron una
variedad de factores específicos y medibles, desde la perspectiva de la Gestión
de Conocimiento la crítica señala la sobre-especificidad de las características
dinámicas y demanda evidencias \cite{Farnese2019,Nissen2019,Schilke2018}. Esto
justifica la necesidad de estudios que organicen los diversos aportes con
diferentes puntos de vista de las áreas conectadas al objeto de estudio,
relacionando conceptos como KD en la ESN, ML y Minería de Procesos
\cite{Pham2021}.

Es preciso analizar estos antecedentes para desarrollar un marco conceptual de
integración tecnológica que establezca alcances y mecanismos, dimensiones y
estructuras de los datos disponibles, roles, normas, comportamientos y modelos
validables hasta el momento \cite{Hacker2021,Namisango_2019}.
Metodológicamente, se procede con la revisión analítica de contenido del estado
actual del conocimiento. No se ha presentado previamente ningún estudio
completo que proporcione una síntesis integral y sistemática con perspectiva en
las capacidades dinámicas de las ESN evidenciando la riqueza de la fuente y en
paralelo que las unifique en un modelo cohesivo y global en relación a KD.

Con los resultados como prueba de concepto se expone el conocimiento que se
puede derivar de las diferentes dimensiones de los estudios y cómo la
combinación de estos puede mejorar la validez del análisis, evidenciar hasta dónde la
investigación ha avanzado y dónde quedan tensiones sin resolver. Con base en
este análisis, sugerir una agenda de investigación que señale el curso para
estudios futuros.

% To print the credit authorship contribution details
% \printcredits

%% Loading bibliography style file
%\bibliographystyle{model1-num-names}
%\bibliographystyle{cas-model2-names}

% Loading bibliography database
%\bibliography{}
\bibliography{cm-sc.bib}
\bibliographystyle{plain}

%% Biography
%\bio{}
%% Here goes the biography details.
%\endbio
%
%\bio{pic1}
%% Here goes the biography details.
%\endbio

\end{document}

