
%%%%%%%%%%%%%%%%%%%%%%% file typeinst.tex %%%%%%%%%%%%%%%%%%%%%%%%%
%
% This is the LaTeX source for the instructions to authors using
% the LaTeX document class 'llncs.cls' for contributions to
% the Lecture Notes in Computer Sciences series.
% http://www.springer.com/lncs       Springer Heidelberg 2006/05/04
%
% It may be used as a template for your own input - copy it
% to a new file with a new name and use it as the basis
% for your article.
%
% NB: the document class 'llncs' has its own and detailed documentation, see
% ftp://ftp.springer.de/data/pubftp/pub/tex/latex/llncs/latex2e/llncsdoc.pdf
%
%%%%%%%%%%%%%%%%%%%%%%%%%%%%%%%%%%%%%%%%%%%%%%%%%%%%%%%%%%%%%%%%%%%


\documentclass[runningheads,a4paper,spanish]{llncs}
% corte de palabras en español
\usepackage[spanish]{babel}
%\usepackage[style=numeric-comp]{biblatex}

\addto{\captionsspanish}{
 \renewcommand{\abstractname}{Resumen}
}

\usepackage{amssymb}
\setcounter{tocdepth}{3}
\usepackage{graphicx}

\usepackage{url}
\usepackage{multirow}

\urldef{\mailsa}\path|{joaquincervino, joseluisgobe, lisandrofernandez}@frba.utn.edu.ar|
\newcommand{\keywords}[1]{\par\addvspace\baselineskip
\noindent\keywordname\enspace\ignorespaces#1}

\begin{document}

\mainmatter  % start of an individual contribution

% first the title is needed
\title{Modelado Conceptual y Ciudades Inteligentes, Un Mapeo Sistemático de Literatura}

% a short form should be given in case it is too long for the running head
\titlerunning{Modelado Conceptual y Ciudades Inteligentes}

% the name(s) of the author(s) follow(s) next
%
% NB: Chinese authors should write their first names(s) in front of
% their surnames. This ensures that the names appear correctly in
% the running heads and the author index.
%
\author{Joaquin Cerviño, Jose Luis Gobe y Lisandro Fernández}
%
\authorrunning{Modelado Conceptual y Ciudades Inteligentes}
% (feature abused for this document to repeat the title also on left hand pages)
% the affiliations are given next; don't give your e-mail address
% unless you accept that it will be published
\institute{
  Universidad Tecnológica Nacional,
  Facultad Regional Buenos Aires\\
  Medrano 951, Buenos Aires, C1179AAQ
  C.A.B.A, Argentina.
  \mailsa\\
  \url{https://www.frba.utn.edu.ar/}
}


% NB: a more complex sample for affiliations and the mapping to the
% corresponding authors can be found in the file "llncs.dem"
% (search for the string "\mainmatter" where a contribution starts).
% "llncs.dem" accompanies the document class "llncs.cls".
%

\toctitle{Modelado Conceptual y Ciudades Inteligentes}
\tocauthor{Un Mapeo Sistemático de Literatura}

\maketitle
\begin{abstract}
The abstract should summarize the contents of the paper and should
contain at least 70 and at most 150 words. It should be written using the
\emph{abstract} environment.
\keywords{We would like to encourage you to list your keywords within
the abstract section}
\end{abstract}


\section{Introducción}


EL modelado conceptual busca representar conceptualizaciones y abstracciones
relevantes del mundo real de tal manera que sea posible apoyar la comunicación,
discusión, análisis y actividades relacionadas \cite{Delcambre2019}.

El modelado conceptual y el razonamiento sobre modelos son capacidades humanas
para observar, comprender e influir en el entorno. A pesar de innumerables
intentos, no existe una definición estricta de uso general de lo que constituye
el modelado conceptual.  Los intentos de definición son variantes de
“Modelado conceptual es modelado con conceptos” e introducir estos conceptos a
través de marcos ontológicos más o menos rígidos, o mediante una explicación
simple usando lenguaje natural \cite{Mawr2020}.

Los modelos conceptuales son modelos de representaciones mentales que agentes
construyen, usan y manipulan durante la actividad cognitiva.  Como tales, no
son modelos de un dominio dado, sino modelos de cómo concebimos ese dominio. Se
los puede caracterizar como artefactos producidos con la intención deliberada
de describir una realidad conceptualizada. De esta forma, se puede afirmar que
establecen contratos de sentido, con el requisito previo de que se exprese una
conexión un modelo que proporcione una semántica conceptual. Estos artefactos
se comprometen con una conceptualización, es decir, la cosmovisión capturada
por dicha conceptualización. En definitiva, se puede afirmar que los modelos
conceptuales captan y comunican un determinado compromiso ontológico
\cite{Guarino2020}.

En la práctica,
La ciudad inteligente
gestiona de manera eficiente los flujos urbanos
a través del proceso en tiempo real de información
sobre dispositivos, ciudadanos y activos. 
% a partir de datos recopilados de diferentes tipos de fuentes,
%\cite{stubinger_understanding_2020}

Las experiencias tempranas de ciudades inteligentes se remontan a la década de
1970, cuando Los Ángeles realiza el primer proyecto de bigdata urbana. Cerca
del comienzo del siglo XXI el interés aumentó significativamente como
consecuencia de la mejora tecnológica y el crecimiento de la poblacion en áreas
urbanas, pero a partir de la década de 2010 es cuando este concepto emerge y se
comienza a discutir \cite{stubinger_understanding_2020}.

Como primeros esfuerzos algunas organizaciones desarrollaron marcos de
evaluación e indicadores de medición. Instituto Británico de Normas (BSI)
estableció un marco de buenas prácticas para la transformación de ciudades
inteligentes y La Organización Internacional de Normalización (ISO) emitió
varias normas de requisitos para el desarrollo de comunidades y ciudades
inteligentes \cite{aljowder_systematic_2019}.

A medida que la ciudad inteligente se despliega, es necesario saber cómo
gestionar y mantener los recursos de los ecosistemas urbanos garantizando la
sustentabilidad económica, social y ambiental general de estas áreas
\cite{aljowder_systematic_2019,stubinger_understanding_2020}.

Apesar el concepto ha sido discutido durante varias decadas, todavía no existe
una definición del término \cite{wahab_systematic_2020}. La ciudad inteligente
es todavía un concepto poco claro sin una nomenclatura estandarizada que pueda
ser efectiva describiéndose a sí mismo.

La gran mayoría de la literatura define “ciudad inteligente” como
infraestructura que cumple las siguientes tres características:
(i) el grupo objetivo son las ciudades y comunidades,
(ii) se mejora la forma de vivir y trabajar en la región,
(iii) se implementan tecnologías de la información y la comunicación (TIC)
\cite{stubinger_understanding_2020}.

De todas formas, la falta de un marco y criterios estandarizados hace que la
mayoría de las ciudades inteligentes basen su desarrollo en un marco
autoregulado. Los involucrados en este proceso no serán capaces de adecuar
correctamente el concepto en sí mismo sin comprender sus fundamentos. Además,
de la necesidad de contar con un marco estandarizado es importante establecer
un plan completo y una concisa comprensión sobre el dominio.

La determinación de un modelo conceptual de ciudad inteligente habilitará a
profesionales, políticos y a la academia a establecer mejores estrategias de
desarrollo y asegurará que que multiples iniciativas estén alineadas.

% estudios
% donde pueda referirse
% como una guía
% para comprender mejor el concepto
.

%% DE WAHAB
% Este concepto se derivó de cinco aspectos diferentes,
% son: 
% las ciudades sostenibles,
% las ciudades inteligentes,
% TIC urbanas,
% desarrollo urbano sostenible,
% sostenibilidad y cuestiones medioambientales,
% y
% urbanización y el crecimiento urbano [7-8 de wahab].


% Por lo tanto, este estudio tiene como objetivo comprender los fundamentos del
% concepto de ciudad inteligente y determinar elementos importantes
% de una ciudad inteligente.


% Las teorías, modelos y conceptos fundamentales en la investigación reflejan
% fenómenos relacionados con la ciudad inteligente.

% Este proceso es crucial
% para responder ya que los estudios interdisciplinarios investigan la ciudad
% inteligente y ven este tema desde diferentes perspectivas. 
% \cite{Anthopoulos2015}.

% Es llamativo que no se presentó previamente ningún estudio académico
% realice una visión general a gran escala basada en enfoques
% cuantitativos/cualitativos que proporcione una síntesis integral y sistemática
% con perspectiva en las capacidades dinámicas de la Ciudades Inteligentes
% evidenciando la riqueza de la fuente y en paralelo las relaciones con un
% marco conceptual cohesivo y global.

% justificacion
% objetivo del articulo

% el objetivo
% el objetivo es presentar los resultados de un mapeo sistematico de la literatura. (systematic mapping study) para establecer el estado del arte sobre tal tema. este sms fue realizado de acuerdo a la directrices propuesta por kitchenham et al y petersen et al
% este sms ha sido realizado segun la directrices propuestas en  ``[kiitchenham yPetersen+
El objetivo del presente artículo es realizar un mapeo sistemático de la
literatura, o SMS por sus siglas en inglés, para establecer el estado del arte
de las contribuciones al modelado conceptual de las Smart Cities. El SMS fue realizado adoptando la metodología descripta por Kitchenham \cite{kitchenham_guidelines_2007}.

% estructura del articulo


%\section{Metodología}\label{metodo}
%
%Metodológicamente, se procede con la revisión analítica de contenido del estado
%actual del conocimiento \cite{kitchenham_guidelines_2007,webster_analyzing_2002,Wolfswinkel2017}.

\section{Planificación del SMS}\label{metodo}

En la presente sección se detalla el protocolo de revisión del SMS:
preguntas de investigación (PI), estrategia de búsqueda, selección de
publicaciones, criterios de inclusión y exclusión, proceso de selección,
estrategia de extracción y síntesis de datos.

El objetivo del SMS es dar respuesta a la pregunta de investigación (PI) :
\textit{¿Cuál es el estado del arte en el modelado conceptual de Smart
Cities?}. Se considera que dicha pregunta principal puede desglosarse en una
serie de subpreguntas para ordenar la búsqueda. Éstas son detalladas a
continuación en la\textbf{Tabla 1}.

% falta una pregunta
\begin{table}[ht]
    \centering
    \begin{tabular}{| c | c | c |}
    \hline
    & Preguntas & Motivación \\ 
    \hline
    \textbf{PI1} & ¿En qué dominios      & Ordenar aportes \\
                 & se realizaron         & de acuerdo a la taxonomía \\ 
                 & contribuciones?       & review de quien era\\
    \hline
    \textbf{PI2} & ¿Qué lenguajes        & Relevar distintos\\
                 & de modelado           & lenguajes de modelado \\ 
                 & se utilizan ?         & ya sea UML otros\\
    \hline
    \textbf{PI3} & ¿Qué tipos de         & Ordenar aportes \\
                 & investigación existen & de acuerdo a la taxonomía \\ 
                 & en los artículos?     & propuesta por Weringa \\
    \hline
%  \textbf{PI2} & ¿En qué dominios se realizaron contribuciones?                                 &            \\ \hline
%  \textbf{PI3} & ¿Qué lenguajes de modelado se utilizan?                                        &            \\ \hline
%  \textbf{PI4} & ¿Qué tipo de investigación existen en los artículos?                           &            \\ \hline
  \end{tabular}
  \end{table}

%Google Scholar
%% Se Enumera más de XXXX estudios en el campo del "Modelado Conceptual de ciudades
%% inteligentes", que van desde modelos teóricos a marcos empíricos.

%% Esto es mio se puede usar
%y la dinámica
%Los escritos en este dominio
%son de  diversidad considerable
%%una revisión sistemática de la literatura
%para obtener el estado
%actual de la investigación,
%identificar, evaluar y sintetizar
%el cuerpo completo de trabajo registrado y producido
%por investigadores, eruditos y practicantes
%es esencial
%%La revisión sistemática de la literatura es
%una revisión sistemática con método explícito, completo y reproducible
%\cite{Okoli2015},
%En base en este análisis,
%señalar tendencias, hallazgos sustantivos
%% sugerir una agenda de investigación que
%e indicar el curso para estudios futuros.


% Con los resultados como prueba de concepto se expone el conocimiento que se
% puede derivar de las diferentes dimensiones de los estudios y cómo la
% combinación de estos puede mejorar la validez del análisis, evidenciar hasta dónde la
% investigación ha avanzado y dónde quedan tensiones sin resolver.
 
% Esta investigación rastrea y analiza la literatura
% disponible sobre los modelos de conceptual de la ciudad inteligente con el
% objetivo de aportar claridad al tema y contribuir a la
% investigación científica sobre el área de estudio.

\section{Desarrollo de Mapeo Sistemático de Literatura}

\subsection{Buscadores}\label{fuentes}

% buscadores
% Fuentes de busqueda

\subsubsection{IEEE}\label{ieee}

https://ieeexplore.ieee.org/Xplorehelp/ieee-xplore-training/user-tips

https://ieeexplore.ieee.org/search/advanced

https://ieeexplore.ieee.org/search

\subsubsection{ACM}\label{acm}

%https://libraries.acm.org/training-resources/new-dl-features/advanced-search-custom-queries

%ttps://dl.acm.org/journals

cadeba de buqueda ACM
((Title: "smart city") OR (Title: "smart cities")) AND ((Title: "conceptual"( OR (Title: "model") OR (Title: "modeling"))

\subsubsection{Scopus}\label{scopus}

%https://tutorials.scopus.com/EN/BasicSearch/sc_BasicSearch_textOnly.html

%https://www.scopus.com/sources?zone=TopNavBar&origin=NO%20ORIGIN%20DEFINED

%https://elsevier.libguides.com/Scopus/topical-search

\subsection{Criterios de Inclusion y exclusion}\label{criterio}

Criterios de Inclusion y exclusion.
% https://books.google.com.ar/books/about/2015_IEEE_First_International_Smart_Citi.html?id=0UEcjwEACAAJ&source=kp_book_description&redir_esc=y

\subsection{Cadenas de busqueda}\label{cadena}

(( "smart city" OR "smart cities") AND ("conceptual model" OR "conceptual modeling"))

\subsection{Criterios de busqueda y exclusion}\label{criterios}

2015 IEEE First International Smart Cities Conference (ISC2)

\subsection{Dimensiones de la Ciudad Inteligente}\label{dimensiones}

% Stubinger habla de categorias
% \cite{stubinger_understanding_2020}

Wahab habla de Dimensions of smart cities \cite{wahab_systematic_2020}
Economy,
Governance,
People,
Environment,
Infrastructure,
Technology,
Living,
Mobility,
Water and Waste,
Security, 
Agriculture
.



\section{Resultado}\label{resultados}
%\section{Sintesis del SMS}\label{resultados}

%\section{Discusión}\label{discusion}
\section{Cosideraciones sobre la validez de este estudio}\label{validez}

%\lipsum[1]

\section{Conclusiones}\label{conclucion}


% \section{BibTeX Entries}
% 
% The correct BibTeX entries for the Lecture Notes in Computer Science
% volumes can be found at the following Website shortly after the
% publication of the book:
% \url{http://www.informatik.uni-trier.de/~ley/db/journals/lncs.html}
% 
% \subsubsection*{Acknowledgments.} The heading should be treated as a
% subsubsection heading and should not be assigned a number.
% 

%\section{The References Section}\label{references}

\bibliography{cm-sc.bib}
\bibliographystyle{plain}

%\begin{thebibliography}{4}

%\end{thebibliography}
 
 
% \section*{Appendix: Springer-Author Discount}
% 
% LNCS authors are entitled to a 33.3\% discount off all Springer
% publications. Before placing an order, the author should send an email, 
% giving full details of his or her Springer publication,
% to \url{orders-HD-individuals@springer.com} to obtain a so-called token. This token is a
% number, which must be entered when placing an order via the Internet, in
% order to obtain the discount.
% 
% \section{Checklist of Items to be Sent to Volume Editors}
% Here is a checklist of everything the volume editor requires from you:
% 
% 
% \begin{itemize}
% \settowidth{\leftmargin}{{\Large$\square$}}\advance\leftmargin\labelsep
% \itemsep8pt\relax
% \renewcommand\labelitemi{{\lower1.5pt\hbox{\Large$\square$}}}
% 
% \item The final \LaTeX{} source files
% \item A final PDF file
% \item A copyright form, signed by one author on behalf of all of the
% authors of the paper.
% \item A readme giving the name and email address of the
% corresponding author.
% \end{itemize}

\end{document}
