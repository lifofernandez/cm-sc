%% 
%% Copyright 2019-2021 Elsevier Ltd
%% 
%% This file is part of the 'CAS Bundle'.
%% --------------------------------------
%% 
%% It may be distributed under the conditions of the LaTeX Project Public
%% License, either version 1.2 of this license or (at your option) any
%% later version.  The latest version of this license is in
%%    http://www.latex-project.org/lppl.txt
%% and version 1.2 or later is part of all distributions of LaTeX
%% version 1999/12/01 or later.
%% 
%% The list of all files belonging to the 'CAS Bundle' is
%% given in the file `manifest.txt'.
%% 
%% Template article for cas-dc documentclass for 
%% double column output.

\documentclass[a4paper,fleqn,spanish]{cas-dc}
% corte de palabras en español
\usepackage[spanish]{babel}
%\usepackage[style=numeric-comp]{biblatex}

%\renewcommand{\abstractname}{Abstract} 
%\renewcommand{\abstract}{Abstract} 
\addto{\captionsspanish}{
  \renewcommand{\abstractname}{ABSTRACT}
}

% If the frontmatter runs over more than one page
% use the longmktitle option.

%\documentclass[a4paper,fleqn,longmktitle]{cas-dc}

%\usepackage[numbers]{natbib}
%\usepackage[authoryear]{natbib}
\usepackage[authoryear,longnamesfirst]{natbib}

\usepackage{lipsum}

% Borrar cover page
\usepackage{atbegshi}% http://ctan.org/pkg/atbegshi
\AtBeginDocument{\AtBeginShipoutNext{\AtBeginShipoutDiscard}}

%%%Author macros
\def\tsc#1{\csdef{#1}{\textsc{\lowercase{#1}}\xspace}}
\tsc{WGM}
\tsc{QE}
%%%

% Uncomment and use as if needed
%\newtheorem{theorem}{Theorem}
%\newtheorem{lemma}[theorem]{Lemma}
%\newdefinition{rmk}{Remark}
%\newproof{pf}{Proof}
%\newproof{pot}{Proof of Theorem \ref{thm}}

\renewcommand{\refname}{Referencias}

\begin{document}

\let\WriteBookmarks\relax
\def\floatpagepagefraction{1}
\def\textpagefraction{.001}

% Short title
\shorttitle{KD in the ESN}    

% Short author
\shortauthors{J. Cerviño, J. L. Gobbe y L. Fernández}  

% Main title of the paper
%\title[mode = title]{Knowledge Discovery in the Enterprise Social Network}    
%\title[mode = title]{Trabajo Práctico Final: Informe del Seminario Integrador}    

\title[mode = title]{Modelado Conceptual de Ciudades Inteligentes}    

% Title footnote mark
% eg: \tnotemark[1]
%\tnotemark[1] 
%
%% Title footnote 1.
%% eg: \tnotetext[1]{Title footnote text}
%\tnotetext[1]{Nota al pie 1} 

% First author
%
% Options: Use if required
% eg: \author[1,3]{Author Name}[type=editor,
%       style=chinese,
%       auid=000,
%       bioid=1,
%       prefix=Sir,
%       orcid=0000-0000-0000-0000,
%       facebook=<facebook id>,
%       twitter=<twitter id>,
%       linkedin=<linkedin id>,
%       gplus=<gplus id>]

%\author[1]{<author name>}[<options>]

\author[1]{Joaquín Cerviño}[
  type=editor,
  style=spanish,
  auid=000,
  bioid=1,
  prefix=Lic.,
  %orcid=0000-0000-0000-0000,
  %facebook=<facebook id>,
  %twitter=<twitter id>,
  %linkedin=<linkedin id>,
  %gplus=<gplus id>
]

% Corresponding author indication
%\cormark[<corr mark no>]
\cormark[1]

% Footnote of the first author
%\fnmark[<footnote mark no>]
%\fnmark[1]

% Email id of the first author
\ead{cjoackin@gmail.com}

% URL of the first author
\ead[https://ioadeer.github.io]{https://ioadeer.github.io}

% Credit authorship
%\credit{Credit authorship details}
% eg: \credit{Conceptualization of this study, Methodology, Software}
\credit{Conceptualization of this study, Methodology, Writing}

\author[2]{José Luis Gobbe}[
  type=editor,
  style=spanish,
  auid=001,
  bioid=2,
  prefix=Ing.,
]
\author[3]{Lisandro Fernández}[
  type=editor,
  style=spanish,
  auid=002,
  bioid=3,
  prefix=Lic.,
]

% Address/affiliation
\affiliation[1-3]{
  organization={
    Universidad Tecnológica Nacional,
    Facultad Regional Buenos Aires
},
  addressline={Medrano 951}, 
  city={Buenos Aires},
  % citysep={}, % Uncomment if no comma needed between city and postcode
  postcode={C1179AAQ}, 
  state={C.A.B.A},
  country={Argentina}
}


% \author[2]{<author name>}[<options>]
% 
% % Footnote of the second author
% \fnmark[2]
% 
% % Email id of the second author
% \ead{}
% 
% % URL of the second author
% \ead[url]{}
% 
% % Credit authorship
% \credit{}
% 
% % Address/affiliation
% \affiliation[<aff no>]{organization={},
%             addressline={}, 
%             city={},
% %          citysep={}, % Uncomment if no comma needed between city and postcode
%             postcode={}, 
%             state={},
%             country={}}
% 

% Corresponding author text
\cortext[1]{Corresponding author}

% Footnote text
% \fntext[1]{Texto nota al pie autor}

% For a title note without a number/mark
\nonumnote{}

\selectlanguage{english}
% Here goes the abstract
\begin{abstract}
The constant increase in the use of the Enterprise Social Network (ESN), as a
channel for organizational collaboration and the large volume of data produced,
provides exceptional opportunities for decision-making in management. The study
of this phenomenon is proposed in order to identify patterns of technological
integration, data structures, techniques, and manipulation mechanisms involved
in the Knowledge Discovery (KD) process in ESNs.
\end{abstract}
\selectlanguage{spanish}

% Use if graphical abstract is present
%\begin{graphicalabstract}
%\includegraphics{}
%\end{graphicalabstract}

% Research highlights
\begin{highlights}
\item pepe
\item pepe
\item pepe
\end{highlights}

%Trabajo Práctico Final: INFORME DEL SEMINARIO INTEGRADOR

% Keywords
% Each keyword is seperated by \sep
\begin{keywords}
Conceptual Model\sep Smart Cities
%\sep
%pepe\sep
\end{keywords}
\maketitle
% Main text
\section{INTRODUCTION}\label{intro}


%\subsection{Título tentativo}\label{titulo-TFE}
%\subsection{Descubrimiento del Conocimiento en la Red Social Empresarial - Un estudio de mapeo sistemático}\label{titulo-TFE}
%\subsection{Objetivo del Trabajo Final de Especialidad}\label{objetivo-TFE}

\lipsum[1-4]


\section{BACKGROUND}\label{marco}

\lipsum[3]

<<<<<<< HEAD

=======
>>>>>>> b43867f4d53af0e401e2a06d45fb559a1e8fd895
%\subsection{Descripción del problema - Identificación de Objeto}\label{que}
\subsection{The smart city concept}\label{concepto}


El incremento constante en el uso de la Red Social Empresarial (Enterprise
Social Network - ESN) \cite{aljowder_systematic_2019} como canal de colaboración
organizacional y el gran volumen de datos producido, brinda oportunidades
excepcionales de estudio para el análisis y la toma de decisiones en la gestión.

El proceso del Descubrimiento del Conocimiento (Knowledge Discovery - KD) en la
ESN trata de extraer patrones en forma de reglas o funciones, a partir de los
datos implicados en la creación e intercambio del contenido generado por el
usuario \cite{wahab_systematic_2020}. Esta tarea involucra analizar y derivar información
nueva de estos aportes, por medio de la identificación de correlaciones entre
términos, logrando explicitar conocimiento mediante la compresión del registro
de eventos que producen los sistemas de colaboración de la organización.

%\subsection{Descripción del problema - Identificación de Objeto}\label{que}
\subsection{ The smart city assessment}\label{afirmacion}

El interés de esta investigación son los patrones de integración tecnológica,
estructuras de datos, técnicas y mecanismos de manipulación involucrados en el
proceso de extraer conocimiento en las ESN \cite{stubinger_understanding_2020}. Este
fenómeno, que se manifiesta en las comunidades online de las organizaciones, es
complejo, multifacético y orbita las tareas de: desarrollar, monitorear y
mejorar procesos; describir objetivos, identificar variables implicadas y
determinar constricciones o posibles alternativas de decisión.

% Numbered list
% Use the style of numbering in square brackets.
% If nothing is used, default style will be taken.
%\begin{enumerate}[a)]
%\item 
%\item 
%\item 
%\end{enumerate}  

% Unnumbered list
%\begin{itemize}
%\item 
%\item 
%\item 
%\end{itemize}  

% Description list
%\begin{description}
%\item[]
%\item[] 
%\item[] 
%\end{description}  

%% % Figure
%% \begin{figure}[<options>]
%% 	\centering
%% 		\includegraphics[<options>]{}
%% 	  \caption{}\label{fig1}
%% \end{figure}
%% 
%% 
%% \begin{table}[<options>]
%% \caption{}\label{tbl1}
%% \begin{tabular*}{\tblwidth}{@{}LL@{}}
%% \toprule
:wahab%%   &  \\ % Table header row
%% \midrule
%%  & \\
%%  & \\
%%  & \\
%%  & \\
%% \bottomrule
%% \end{tabular*}
%% \end{table}

% Uncomment and use as the case may be
%\begin{theorem} 
%\end{theorem}

% Uncomment and use as the case may be
%\begin{lemma} 
%\end{lemma}

%% The Appendices part is started with the command \appendix;
%% appendix sections are then done as normal sections
%% \appendix

\section{RESEARCH METHOD }\label{metodo}

La disponibilidad, facilidad de uso y escalabilidad de sistemas lo
suficientemente robustos para manejar instancias a gran escala de registro de
datos estructurados y no estructurados de múltiples fuentes ha creado entornos
de trabajo fácil, colaborativo y distributivo
\cite{wieringa_requirements_2006, kitchenham_guidelines_2007, webster_analyzing_2002}.

\subsection{ Inclusion and exclusion criteria}\label{criterio}

\lipsum[2]

<<<<<<< HEAD

\subsection{ Determine search sources}\label{fuentes}

\lipsum[1]


\subsection{Define search string}\label{cadena}

\lipsum[3]


\subsection{Search and selection}\label{seleccion}

\lipsum[4-5]


\section{IV. RESULTS ANALYSIS}\label{resultados}

\lipsum[2]


\subsection{Smart city and conceptual models identified from the scientific literature}\label{sci-lit}

\lipsum[3-4]

\subsection{B. The regions/countries covered by the related studies}\label{regiones}

\lipsum[3-4]


\subsection{C. The frameworks main components}\label{componentes}

\lipsum[3-4]


\subsection{Smart city alignment with the Sustainable Development Goals (SDGs)}\label{sustentabilidad}

\lipsum[2-4]

\section{V. DISCUSSION}\label{discusion}

\lipsum[1]


\section{VI. CONCLUSION}\label{conclucion}

\lipsum[2-4]

=======
\subsection{ Determine search sources}\label{fuentes}

Lorem ipsum dolor sit amet, consectetur adipiscing elit. Nullam facilisis urna
id ligula elementum suscipit. Donec lobortis iaculis dolor sit amet tincidunt.
Pellentesque eu finibus libero, vel blandit purus. Ut egestas vitae lectus ac
ornare. Mauris volutpat velit non nulla lacinia, eu interdum orci tincidunt.
Nullam ac fringilla odio. Aliquam dictum nisl a ex hendrerit, sed dictum diam
hendrerit. Integer facilisis vulputate vulputate. Vivamus vitae dignissim sem.
Ut porta ante vitae arcu dignissim tempus.

\subsection{Define search string}\label{cadena}

Lorem ipsum dolor sit amet, consectetur adipiscing elit. Nullam facilisis urna
id ligula elementum suscipit. Donec lobortis iaculis dolor sit amet tincidunt.
Pellentesque eu finibus libero, vel blandit purus. Ut egestas vitae lectus ac
ornare. Mauris volutpat velit non nulla lacinia, eu interdum orci tincidunt.
Nullam ac fringilla odio. Aliquam dictum nisl a ex hendrerit, sed dictum diam
hendrerit. Integer facilisis vulputate vulputate. Vivamus vitae dignissim sem.
Ut porta ante vitae arcu dignissim tempus.


\subsection{Search and selection}\label{seleccion}

Lorem ipsum dolor sit amet, consectetur adipiscing elit. Nullam facilisis urna
id ligula elementum suscipit. Donec lobortis iaculis dolor sit amet tincidunt.
Pellentesque eu finibus libero, vel blandit purus. Ut egestas vitae lectus ac
ornare. Mauris volutpat velit non nulla lacinia, eu interdum orci tincidunt.
Nullam ac fringilla odio. Aliquam dictum nisl a ex hendrerit, sed dictum diam
hendrerit. Integer facilisis vulputate vulputate. Vivamus vitae dignissim sem.
Ut porta ante vitae arcu dignissim tempus.


\section{IV. RESULTS ANALYSIS}\label{resultados}

Lorem ipsum dolor sit amet, consectetur adipiscing elit. Nullam facilisis urna
id ligula elementum suscipit. Donec lobortis iaculis dolor sit amet tincidunt.
Pellentesque eu finibus libero, vel blandit purus. Ut egestas vitae lectus ac
ornare. Mauris volutpat velit non nulla lacinia, eu interdum orci tincidunt.
Nullam ac fringilla odio. Aliquam dictum nisl a ex hendrerit, sed dictum diam
hendrerit. Integer facilisis vulputate vulputate. Vivamus vitae dignissim sem.
Ut porta ante vitae arcu dignissim tempus.


\subsection{Smart city and conceptual models identified from the scientific literature}\label{sci-lit}

Cras est elit, finibus vel placerat at, convallis a neque. Maecenas nibh nisi,
sagittis eget luctus eget, consequat in ligula. Nulla efficitur enim ut felis
efficitur vestibulum venenatis quis velit. Nunc sapien felis, consectetur et
libero nec, posuere efficitur lectus. Praesent posuere purus nec metus
fermentum, at luctus sapien vehicula. Praesent at pellentesque tellus. Duis eget
bibendum leo. Curabitur finibus varius rhoncus.

Etiam molestie, nisl eu cursus malesuada, ante diam pulvinar lectus, ut porta
libero justo eget nisi. Proin posuere odio id lorem accumsan, id feugiat diam
pretium. Ut sed eleifend mauris, ac cursus massa. Sed maximus sem quam, non
volutpat ipsum condimentum quis. In vestibulum congue aliquet. Duis sit amet
sapien blandit, scelerisque quam consequat, imperdiet dolor. Nunc erat ligula,
pharetra id tellus nec, suscipit varius augue.


\subsection{B. The regions/countries covered by the related studies}\label{regiones}

Cras est elit, finibus vel placerat at, convallis a neque. Maecenas nibh nisi,
sagittis eget luctus eget, consequat in ligula. Nulla efficitur enim ut felis
efficitur vestibulum venenatis quis velit. Nunc sapien felis, consectetur et
libero nec, posuere efficitur lectus. Praesent posuere purus nec metus
fermentum, at luctus sapien vehicula. Praesent at pellentesque tellus. Duis eget
bibendum leo. Curabitur finibus varius rhoncus.

Etiam molestie, nisl eu cursus malesuada, ante diam pulvinar lectus, ut porta
libero justo eget nisi. Proin posuere odio id lorem accumsan, id feugiat diam
pretium. Ut sed eleifend mauris, ac cursus massa. Sed maximus sem quam, non
volutpat ipsum condimentum quis. In vestibulum congue aliquet. Duis sit amet
sapien blandit, scelerisque quam consequat, imperdiet dolor. Nunc erat ligula,
pharetra id tellus nec, suscipit varius augue.


\subsection{C. The frameworks main components}\label{componentes}

Cras est elit, finibus vel placerat at, convallis a neque. Maecenas nibh nisi,
sagittis eget luctus eget, consequat in ligula. Nulla efficitur enim ut felis
efficitur vestibulum venenatis quis velit. Nunc sapien felis, consectetur et
libero nec, posuere efficitur lectus. Praesent posuere purus nec metus
fermentum, at luctus sapien vehicula. Praesent at pellentesque tellus. Duis eget
bibendum leo. Curabitur finibus varius rhoncus.

Etiam molestie, nisl eu cursus malesuada, ante diam pulvinar lectus, ut porta
libero justo eget nisi. Proin posuere odio id lorem accumsan, id feugiat diam
pretium. Ut sed eleifend mauris, ac cursus massa. Sed maximus sem quam, non
volutpat ipsum condimentum quis. In vestibulum congue aliquet. Duis sit amet
sapien blandit, scelerisque quam consequat, imperdiet dolor. Nunc erat ligula,
pharetra id tellus nec, suscipit varius augue.


\subsection{Smart city alignment with the Sustainable Development Goals (SDGs)}\label{sustentabilidad}

Cras est elit, finibus vel placerat at, convallis a neque. Maecenas nibh nisi,
sagittis eget luctus eget, consequat in ligula. Nulla efficitur enim ut felis
efficitur vestibulum venenatis quis velit. Nunc sapien felis, consectetur et
libero nec, posuere efficitur lectus. Praesent posuere purus nec metus
fermentum, at luctus sapien vehicula. Praesent at pellentesque tellus. Duis eget
bibendum leo. Curabitur finibus varius rhoncus.

Etiam molestie, nisl eu cursus malesuada, ante diam pulvinar lectus, ut porta
libero justo eget nisi. Proin posuere odio id lorem accumsan, id feugiat diam
pretium. Ut sed eleifend mauris, ac cursus massa. Sed maximus sem quam, non
volutpat ipsum condimentum quis. In vestibulum congue aliquet. Duis sit amet
sapien blandit, scelerisque quam consequat, imperdiet dolor. Nunc erat ligula,
pharetra id tellus nec, suscipit varius augue.


\section{V. DISCUSSION}\label{discusion}

Cras est elit, finibus vel placerat at, convallis a neque. Maecenas nibh nisi,
sagittis eget luctus eget, consequat in ligula. Nulla efficitur enim ut felis
efficitur vestibulum venenatis quis velit. Nunc sapien felis, consectetur et
libero nec, posuere efficitur lectus. Praesent posuere purus nec metus
fermentum, at luctus sapien vehicula. Praesent at pellentesque tellus. Duis eget
bibendum leo. Curabitur finibus varius rhoncus.

Etiam molestie, nisl eu cursus malesuada, ante diam pulvinar lectus, ut porta
libero justo eget nisi. Proin posuere odio id lorem accumsan, id feugiat diam
pretium. Ut sed eleifend mauris, ac cursus massa. Sed maximus sem quam, non
volutpat ipsum condimentum quis. In vestibulum congue aliquet. Duis sit amet
sapien blandit, scelerisque quam consequat, imperdiet dolor. Nunc erat ligula,
pharetra id tellus nec, suscipit varius augue.


\section{VI. CONCLUSION}\label{conclucion}

Cras est elit, finibus vel placerat at, convallis a neque. Maecenas nibh nisi,
sagittis eget luctus eget, consequat in ligula. Nulla efficitur enim ut felis
efficitur vestibulum venenatis quis velit. Nunc sapien felis, consectetur et
libero nec, posuere efficitur lectus. Praesent posuere purus nec metus
fermentum, at luctus sapien vehicula. Praesent at pellentesque tellus. Duis eget
bibendum leo. Curabitur finibus varius rhoncus.

Etiam molestie, nisl eu cursus malesuada, ante diam pulvinar lectus, ut porta
libero justo eget nisi. Proin posuere odio id lorem accumsan, id feugiat diam
pretium. Ut sed eleifend mauris, ac cursus massa. Sed maximus sem quam, non
volutpat ipsum condimentum quis. In vestibulum congue aliquet. Duis sit amet
sapien blandit, scelerisque quam consequat, imperdiet dolor. Nunc erat ligula,
pharetra id tellus nec, suscipit varius augue.
>>>>>>> b43867f4d53af0e401e2a06d45fb559a1e8fd895



% To print the credit authorship contribution details
% \printcredits

%% Loading bibliography style file
%\bibliographystyle{model1-num-names}
%\bibliographystyle{cas-model2-names}

% Loading bibliography database
%\bibliography{}
\bibliography{cm-sc.bib}
\bibliographystyle{plain}

%% Biography
%\bio{}
%% Here goes the biography details.
%\endbio
%
%\bio{pic1}
%% Here goes the biography details.
%\endbio

\end{document}

