%% 
 % Copyright 2019-2021 Elsevier Ltd
 % 
 % This file is part of the 'CAS
%Bundle'.
 % --------------------------------------
 % 
 % It may be distributed
%under the conditions of the LaTeX Project Public
 % License, either version 1.2
%of this license or (at your option) any
 % later version. The latest version of
%this license is in
 % http://www.latex-project.org/lppl.txt
 % and version 1.2
%or later is part of all distributions of LaTeX
 % version 1999/12/01 or later.
%% 
 % The list of all files belonging to the 'CAS Bundle' is
 % given in the
%file `manifest.txt'.
 % 
 % Template article for cas-dc documentclass for 
 %
%double column output.

\documentclass[a4paper,fleqn,spanish]{cas-dc}
% corte de palabras en español
\usepackage[spanish]{babel}
%\usepackage[style=numeric-comp]{biblatex}

%\renewcommand{\abstractname}{Abstract} 
 \renewcommand{\abstract}{Abstract} 
\addto{\captionsspanish}{
 \renewcommand{\abstractname}{ABSTRACT}
 }

% If the frontmatter runs over more than one page
 use the longmktitle option.

%\documentclass[a4paper,fleqn,longmktitle]{cas-dc}

%\usepackage[numbers]{natbib}
 \usepackage[authoryear]{natbib}
\usepackage[authoryear,longnamesfirst]{natbib}

\usepackage{lipsum}

% Borrar cover page
\usepackage{atbegshi}% http://ctan.org/pkg/atbegshi
\AtBeginDocument{\AtBeginShipoutNext{\AtBeginShipoutDiscard}}

%%%Author macros
\def\tsc#1{\csdef{#1}{\textsc{\lowercase{#1}}\xspace}}
 \tsc{WGM}
 \tsc{QE}
%%%

% Uncomment and use as if needed
 \newtheorem{theorem}{Theorem}
% \newtheorem{lemma}[theorem]{Lemma}
 \newdefinition{rmk}{Remark}
% \newproof{pf}{Proof}
 \newproof{pot}{Proof of Theorem \ref{thm}}

\renewcommand{\refname}{Referencias}

\begin{document}

\let\WriteBookmarks\relax
 \def\floatpagepagefraction{1}
\def\textpagefraction{.001}

% Short title
\shorttitle{KD in the ESN} 

% Short author
\shortauthors{J. Cerviño, J. L. Gobbe y L. Fernández} 

% Main title of the paper
 \title[mode = title]{Knowledge Discovery in the
% Enterprise Social Network} 
 \title[mode = title]{Trabajo Práctico Final:
% Informe del Seminario Integrador} 

\title[mode = title]{Modelado Conceptual de Ciudades Inteligentes} 

% Title footnote mark
 eg: \tnotemark[1]
 \tnotemark[1] 
% 
% % Title footnote 1.
 % eg: \tnotetext[1]{Title footnote text}
% \tnotetext[1]{Nota al pie 1} 

% First author
 \author[1]{<author name>}[<options>]
% 
% Options: Use if required
 eg: \author[1,3]{Author Name}[type=editor,
% style=chinese,
 auid=000,
 bioid=1,
 prefix=Sir,
 orcid=0000-0000-0000-0000,
% facebook=<facebook id>,
 twitter=<twitter id>,
 linkedin=<linkedin id>,

\author[1]{Joaquín Cerviño}[
 type=editor,
 style=spanish,
 auid=000,
 bioid=1,
prefix=Lic.,
%orcid=0000-0000-0000-0000,
 facebook=<facebook id>,
 twitter=<twitter id>,
%linkedin=<linkedin id>,
 gplus=<gplus id>
]

% Corresponding author indication
 \cormark[<corr mark no>]
\cormark[1]

% Footnote of the first author
 \fnmark[<footnote mark no>]
 \fnmark[1]

% Email id of the first author
\ead{cjoackin@gmail.com}

% URL of the first author
\ead[https://ioadeer.github.io]{https://ioadeer.github.io}

% Credit authorship
 \credit{Credit authorship details}
 eg:
% \credit{Conceptualization of this study, Methodology, Software}
\credit{Conceptualization of this study, Methodology, Writing}

\author[2]{José Luis Gobbe}[
 type=editor,
 style=spanish,
 auid=001,
 bioid=2,
prefix=Ing.,
 ]
 \author[3]{Lisandro Fernández}[
 type=editor,
 style=spanish,
auid=002,
 bioid=3,
 prefix=Lic.,
 ]

% Address/affiliation
\affiliation[1-3]{
 organization={
 Universidad Tecnológica Nacional,
 Facultad
Regional Buenos Aires
 },
 addressline={Medrano 951}, 
 city={Buenos Aires},
% citysep={}, % Uncomment if no comma needed between city and postcode
postcode={C1179AAQ}, 
 state={C.A.B.A},
 country={Argentina}
 }


% \author[2]{<author name>}[<options>]
% 
% % Footnote of the second author
 \fnmark[2]
% 
% % Email id of the second author
 \ead{}
% 
% % URL of the second author
 \ead[url]{}
% 
% % Credit authorship
 \credit{}
% 
% % Address/affiliation
 \affiliation[<aff no>]{organization={},
% addressline={}, 
 city={},
 %   citysep={}, % Uncomment if no comma needed
% between city and postcode
 postcode={}, 
 state={},
 country={}}
% 

% Corresponding author text
\cortext[1]{Corresponding author}

% Footnote text
\fntext[1]{Texto nota al pie autor}

% For a title note without a number/mark
\nonumnote{}

\selectlanguage{english}

\begin{abstract}
Etiam euismod. Fusce facilisis lacinia dui. Suspendisse potenti. In mi erat,
cursus id, nonummy sed, ullamcorper eget, sapien. Praesent pretium, magna in
eleifend egestas, pede pede pretium lorem, quis consectetuer tortor sapien
facilisis magna. Mauris quis magna varius nulla scelerisque imperdiet. Aliquam
non quam. Aliquam porttitor quam a la- cus. Praesent vel arcu ut tortor cursus
volutpat. In vitae pede quis diam bibendum placerat.
\end{abstract}

\selectlanguage{spanish}

% Use if graphical abstract is present
\begin{graphicalabstract}
% \includegraphics{}
\end{graphicalabstract}

% Research highlights
\begin{highlights}
 \item pepe
 \item pepe
 \item pepe
\end{highlights}


% Keywords
 Each keyword is seperated by \sep
\begin{keywords}
 Conceptual Model\sep Smart Cities
 \end{keywords}

\maketitle

% Main text
\section{INTRODUCTION}\label{intro}
% Objetivo del Trabajo Final de Especialidad
\lipsum[12]

\section{BACKGROUND}\label{marco}
%% Descripción del problema - Identificación de Objeto

% Aca breve introduccion al los aparatados 
% modelado conceptual y la ciudad
% inteligente.

%Según Gartner,
%el líder mundial empresa de
%investigación y asesoramiento,

En la practica, la ciudad inteligente procesa información sobre edificios,
ciudadanos, dispositivos y activos a partir de datos recopilados de diferentes
tipos de fuentes, para gestionar de manera eficiente los flujos urbanos a
través de respuestas en tiempo real \cite{stubinger_understanding_2010}. 

%\subsection{The smart city concept}\label{concepto}
\subsection{La Ciudad Inteligente}\label{concepto}

Las experiencias tempranas de ciudades inteligentes ocurren durante la década
de 1970, cuando Los Ángeles realiza el primer proyecto de datos urbanos a
gran escala \cite{stubinger_understanding_2010}. Desde principio del siglo XXI,
el interés ha aumentado significativamente como consecuencia de la mejora
tecnológica así como la poblacion en áreas urbanas, pero recien en la
década de 2010 este concepto emergió y se discutió a fondo. A medida que
avanza la implementación de la ciudad inteligente, es necesario saber cómo
gestionar y mantener los recursos de los ecosistemas urbanos
\cite{aljowder_systematic_2010} garantizando la sostenibilidad económica,
social y ambiental general \cite{stubinger_understanding_2010} de estas areas.

% unir y acortar estos 3 parrafos
Aunque el concepto ha sido introducido y discutido desde hace algunos años,
investigadores coinciden en que todavía no existe una definición del término
ciudad inteligente [6, 9 a 12 de wahab]. La ciudad inteligente es todavía un
concepto poco claro sin una nomenclatura estandarizada que pueda ser
efectivamente describiéndose a sí mismo.

Todavía falta un marco y criterios estandarizados que contribuye a una ciudad
inteligente, lo que hace que la mayoría de las ciudades inteligentes que se
están desarrollando se basaron en marco auto-regulado.
Los involucrados en el desarrollo de las ciudades inteligentes no
serian capaces de desarrollar correctamente el concepto de ciudad inteligente en
sí mismo sin comprender los fundamentos del mismo.
Además, existe la necesidad de contar con un marco estandarizado para poder
utilizarlo como guía para proyectos de ciudades inteligentes en el futuro.  Es
importante establecer un plan completo y conciso comprensión sobre el concepto
de ciudad inteligente, ya que sirve como un terreno común de lo que se trata la
ciudad inteligente.  Facilitará a profesionales, políticos y la academia tener
una mejor comprensión del concepto y asegurar que las iniciativas que se están
realizando estén en línea con el concepto, así como establecer mejores
estrategias para llevar a cabo las iniciativas.

%% DE WAHAB
% Este concepto se derivó de cinco aspectos diferentes,
% son: 
% las ciudades sostenibles,
% las ciudades inteligentes,
% TIC urbanas,
% desarrollo urbano sostenible,
% sostenibilidad y cuestiones medioambientales,
% y
% urbanización y el crecimiento urbano [7-8 de wahab].

La gran mayoría de la literatura define la término “ciudad inteligente” como
aplicaciones y tecnologías que cumplen las siguientes tres características:
(i) El grupo objetivo son las ciudades y comunidades,
(ii) se mejora la forma de vivir y trabajar en la región,
(iii) se implementan las tecnologías de la información y la comunicación (TIC)
\cite{stubinger_understanding_2020}.

Como primeros esfuerzos, organizaciones internacionales y regionales
proporcionan marcos de evaluación e indicadores de medición. Como por ejemplo, la
Organización Internacional de Normalización (ISO) emitió varias normas
relacionadas con los requisitos de desarrollo sostenible en comunidades y
ciudades inteligentes infraestructura y el Instituto Británico de Normas
(BSI) estableció “PAS 181”, que es un marco de buenas prácticas para la
transformación de ciudades inteligentes \cite{aljowder_systematic_2019}.

% Por lo tanto, este estudio tiene como objetivo comprender los fundamentos del
% concepto de ciudad inteligente y también determinar los elementos importantes
% de una ciudad inteligente.

La determinación de la ciudad inteligente ayudará en el desarrollo de un modelo
conceptual donde futuros estudios podrán referirse como una guía para
comprender mejor el concepto \cite{wahab_systematic_2020}.

% Las teorías, modelos y conceptos fundamentales en la investigación reflejan
% fenómenos relacionados con la ciudad inteligente.
% Este proceso es crucial
% para responder ya que los estudios interdisciplinarios investigan la ciudad
% inteligente y ven este tema desde diferentes perspectivas. 
% \cite{Anthopoulos2015}.


\subsection{Modelado Conceptual}\label{afirmacion}


\section{RESEARCH METHOD}\label{metodo}

%% ESto es mio se puede usar
Metodológicamente, se procede con la revisión analítica de contenido del estado
actual del conocimiento \cite{kitchenham_guidelines_2007,
webster_analyzing_2002} siguiendo las pautas propuestas por
\cite{Wolfswinkel2017}.

La revisión sistemática de la literatura es “una revisión sistemática, método
explícito, completo y reproducible para identificar, evaluar y sintetizar el
cuerpo existente de trabajo completo y grabado producido por investigadores,
eruditos y practicantes” \cite{Okoli2015}.

Es sorprendente que no se presento previamente ningún estudio académico
 realice
una visión general a gran escala basada en enfoques
 cuantitativos/cualitativos
que proporcione una síntesis integral y sistemática
 con perspectiva en las
capacidades dinámicas de la Ciudades Inteligentes
 evidenciando la riqueza de la
fuente y en paralelo las relaciones con un
 Modelo Conceptual cohesivo y
global.

%Google Scholar
Se Enumera más de XXXX estudios en el campo del "Modelado Conceptual de
ciudades
 inteligentes", que van desde modelos teóricos a marcos empíricos.
%% ESto es mio se puede usar
Considerando la diversidad y la dinámica de los manuscritos en este dominio,
una revisión sistemática de la literatura es esencial para obtener el estado
actual de la investigación, incluyendo hallazgos sustantivos, tendencias y en
base en este análisis,
% sugerir una agenda de investigación que
señalar el curso para estudios futuros.

% Con los resultados como prueba de concepto se expone el conocimiento que se
% puede derivar de las diferentes dimensiones de los estudios y cómo la
% combinación de estos puede mejorar la validez del análisis, evidenciar hasta
% dónde la
 investigación ha avanzado y dónde quedan tensiones sin resolver.
% 
% Esta investigación rastreará y analizará la literatura
 disponible sobre los
% modelos de conceptual de la ciudad inteligente con el
 objetivo de aportar
% claridad al tema y contribuir a la
 investigación científica sobre el área de
% estudio.

\subsection{Inclusion and exclusion criteria}\label{criterio}

Criterios de Inclusion y exclusion.

\subsection{Determine search sources}\label{fuentes}

buscadores
 Fuentes de busqueda


\subsection{Define search string}\label{cadena}

Cadena de Busqueda.


\subsection{Search and selection}\label{seleccion}

Busqueda y seleccion.


\section{RESULTS ANALYSIS ó Categorization of Smart City Literature ó
Findings}\label{resultados}

Wahab habla de Dimensions of smart cities \cite{wahab_systematic_2020}

Stubinger habla de categorias \cite{stubinger_understanding_2020}
 Economy,
Governance,
 People,
 Environment,
 Infrastructure,
 Technology,
 Living,
Mobility,
 Water and Waste,
 Security, 
 Agriculture.


\subsection{Conceptual model of smart cities identified from the scientific
literature}\label{sci-lit}

%\lipsum[3-4]

\subsection{The regions/countries covered by the related
studies}\label{regiones}

%\lipsum[3-4]


\subsection{The frameworks main components}\label{componentes}

%\lipsum[3-4]


\subsection{Smart city alignment with the Sustainable Development Goals
(SDGs)}\label{sustentabilidad}

%\lipsum[2-4]

\section{DISCUSSION}\label{discusion}

%\lipsum[1]


\section{CONCLUSION}\label{conclucion}

%\lipsum[2-4]



% To print the credit authorship contribution details
 %\printcredits

%% Loading bibliography style file
 \bibliographystyle{model1-num-names}
%\bibliographystyle{cas-model2-names}

% Loading bibliography database
 \bibliography{}
\bibliography{cm-sc.bib}
 \bibliographystyle{plain}

%% Biography
%\bio{}
%% Here goes the biography details.
%\endbio
%
%\bio{pic1}
 % Here goes the biography details.
 %\endbio

\end{document}


%% % Numbered list
%%  Use the style of numbering in square brackets.
%%  If nothing is
%% % used, default style will be taken.
%%  \begin{enumerate}[a)]
%%  \item 
%%  \item 
%% % \item 
%%  \end{enumerate} 
%% 
%% % Unnumbered list
%%  \begin{itemize}
%%  \item 
%%  \item 
%%  \item 
%%  \end{itemize} 
%% 
%% % Description list
%%  \begin{description}
%%  \item[]
%%  \item[] 
%%  \item[] 
%% \end{description} 
%% 
%% %% % Figure
%%  % \begin{figure}[<options>]
%%  % 	\centering
%%  %
%% %\includegraphics[<options>]{}
%%  % 	 \caption{}\label{fig1}
%%  % \end{figure}
%%  % 
%% %% 
%%  % \begin{table}[<options>]
%%  % \caption{}\label{tbl1}
%%  %
%% %\begin{tabular*}{\tblwidth}{@{}LL@{}}
%%  % \toprule
%%  % & \\ % Table header row
%%  %
%% %\midrule
%%  % & \\
%%  % & \\
%%  % & \\
%%  % & \\
%%  % \bottomrule
%%  % \end{tabular*}
%%  %
%% %\end{table}
%% 
%% % Uncomment and use as the case may be
%%  \begin{theorem} 
%%  \end{theorem}
%% 
%% % Uncomment and use as the case may be
%%  \begin{lemma} 
%%  \end{lemma}
%% 
%% %% The Appendices part is started with the command \appendix;
%%  % appendix
%% %sections are then done as normal sections
%%  % \appendix